\section{Chart::HorizontalBars}
\herv{Name:} Chart::HorizontalBars\\ \\
\herv{File:} HorizontalBars.pm\\ \\
\herv{Requires:}Chart::Base, GD, Carp, FileHandle\\ \\
\herv{Description:} \fett{HorizontalBars} is a \fett{subclass} of Chart::Base.\\
The class HorizontalBars creates a chart with bars, that run horizontal.\\
\\
\herv{Example:}
\begin{figure}[h]
	\begin{center}
		\includegraphics[scale=0.7]{d_hbars4.png}
	\end{center}
	\caption{Chart with horizontal bars}
	\label{fig:hbars}
\end{figure}
\begin{verbatim}
use Chart::HorizontalBars;

$g = Chart::HorizontalBars->new();
$g->add_dataset ('Foo', 'bar', 'junk', 'ding', 'bat');
$g->add_dataset (4, 3, 4, 2, 8);
$g->add_dataset (2, 10, 3, 8, 3);

%hash = ( 'title' => 'Horizontal Bars Demo',
          'grid_lines' => 'true',
          'x_label' => 'x-axis',
          'y_label' => 'y-axis',
          'include_zero' => 'true',
          'x_ticks' => 'vertical',
         );
$g->set (%hash);

$g->png ("hbars.png");
\end{verbatim}
\herv{Constructor:} An instance of a HorizontalBars object can be created with the constructor new():\\
\fett{\$obj = Chart::HorizontalBars->new();}\\
\fett{\$obj = Chart::HorizontalBars->new(\kursiv{width}, \kursiv{height});}\\
\\
If \fett{new} has no arguments, the constructor returns an image with the size 300x400 pixels. If new has two arguments \kursiv{width} and \kursiv{height}, it returns an image with the desired size. \\ 
\\ 
\herv{Methods:}All universally valid methods, see page \pageref{methods}: Chart::Base. \\
\\
\herv{Attributes/Options:} All universally valid options, see page \pageref{options}. Also available, these special options:
\begin{description}
\item['y\_axes'] Tells chart where to place the y-axis. Valid values are 'left', 'right' and 'both'. Defaults to 'left'.
\item['spaced\_bars']Leaves space between the groups of bars at each data point when set to 'true'.  This just makes it easier to read a bar chart.  Default is 'true'.
\item['skip\_y\_ticks'] Does the same for the y-axis at a HorizontalBars chart as 'skip\_x\_ticks' does for other charts. Defaults to 1.
\end{description}